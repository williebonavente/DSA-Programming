% Set the Page Layout
\documentclass[12pt]{article}
\usepackage[inner = 2.0cm, outer = 2.0cm, top = 2.0cm, bottom = 2.0cm]{geometry}


% Package to write pseudo-codes
\usepackage{algorithm}

% Remove the 'end' at the end of the algorithm
\usepackage[noend]{algpseudocode}

% Define Left Justified Comments
\algnewcommand{\LeftComment}[1]{\Statex \(\triangleright\) #1}

% Remove the Numbering of the Algorithm
\usepackage{caption}
\DeclareCaptionLabelFormat{algnonumber}{Algorithm}
\captionsetup[algorithm]{labelformat = algnonumber}

\begin{document}

\begin{algorithm}

  \caption{Prints the elements in the level order fashion \textbf{Line By Line}}
  \begin{algorithmic}[1]
    \Require A non empty tree
    \Ensure Each level is printed in a different line and one queue is used
    \Statex
    \Function{Level\_Order\_Traversal}{$root$}
        \LeftComment Queue stores the pointers of the nodes.
        
        \Statex
        \State $marker \gets nullptr$
        \State $queue.push(root)$
        \State $queue.push(marker)$
        \While{Queue is \textbf{not empty, }}
            \If{$queue.front == marker$}
                \State \textbf{Print}(One Level has been finished)
                \State $queue.pop$
                \If{$queue$ is \textbf{not empty}}
                    \State $queue.push(marker)$
                \EndIf
            \EndIf
            
            \State $current\_node\gets queue.front $
            \State $queue.pop$
            \State \textbf{Print}($current\_node.data$)
            
            \If{Left Child Exists}
                \State $queue.push(current\_node.left)$
            \EndIf
            
            \If{Right Child Exists}
                \State $queue.push(current\_node.right)$
            \EndIf
            
        \EndWhile
        \Statex
        \State \textbf{Print}(Level Order Traversal finished)
    \EndFunction
  \end{algorithmic}
  
\end{algorithm}

\end{document}
