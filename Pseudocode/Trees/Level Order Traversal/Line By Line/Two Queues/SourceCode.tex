% Set the Page Layout
\documentclass[12pt]{article}
\usepackage[inner = 2.0cm, outer = 2.0cm, top = 2.0cm, bottom = 2.0cm]{geometry}


% Package to write pseudo-codes
\usepackage{algorithm}

% Remove the 'end' at the end of the algorithm
\usepackage[noend]{algpseudocode}

% Define Left Justified Comments
\algnewcommand{\LeftComment}[1]{\Statex \(\triangleright\) #1}

% Remove the Numbering of the Algorithm
\usepackage{caption}
\DeclareCaptionLabelFormat{algnonumber}{Algorithm}
\captionsetup[algorithm]{labelformat = algnonumber}

\begin{document}

\begin{algorithm}

  \caption{Prints the elements in the level order fashion \textbf{Line By Line}}
  \begin{algorithmic}[1]
    \Require A non empty tree
    \Ensure Each level is printed in a different line 
    \Statex
    \Function{Level\_Order\_Traversal}{$root$}
        \LeftComment Queue stores the pointers of the nodes.
        
        \Statex
        \State $main\_queue.push(root)$
        
        \While{Main Queue is \textbf{not empty, }}
            \State \textbf{Print}(Here comes a new level)
            \While{Main Queue is \textbf{not empty, }}
                \State $current\_node\gets main\_queue.front $
                \State $queue.pop$
                \State \textbf{Print}($current\_node.data$)
                
                %\Statex
                \If{Left Child Exists}
                    \State $aux\_queue.push(current\_node.left)$
                \EndIf
                
                \If{Right Child Exists}
                    \State $aux\_queue.push(current\_node.right)$
                \EndIf
            \EndWhile
            \State \textbf{Print}(This level has been printed)
            \State \textbf{Swap}($main\_queue, aux\_queue)$
        \EndWhile
        \Statex
        \State \textbf{Print}(Level Order Traversal finished)
    \EndFunction
  \end{algorithmic}
  
\end{algorithm}

\end{document}
