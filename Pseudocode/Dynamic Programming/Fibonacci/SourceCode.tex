% Set the Page Layout
\documentclass[12pt]{article}
\usepackage[inner = 2.0cm, outer = 2.0cm, top = 2.0cm, bottom = 2.0cm]{geometry}


% Package to write pseudo-codes
\usepackage{algorithm}

% Remove the 'end' at the end of the algorithm
\usepackage[noend]{algpseudocode}

% Define Left Justified Comments
\algnewcommand{\LeftComment}[1]{\Statex \(\triangleright\) #1}

% Remove the Numbering of the Algorithm
\usepackage{caption}
\DeclareCaptionLabelFormat{algnonumber}{Algorithm}
\captionsetup[algorithm]{labelformat = algnonumber}

% Define the command for a boldface instructions
\newcommand{\Is}{\textbf{ is }}
\newcommand{\To}{\textbf{ to }}
\newcommand{\Downto}{\textbf{ downto }}
\newcommand{\Or}{\textbf{ or }}
\newcommand{\And}{\textbf{ and }}

\begin{document}

\begin{algorithm}

  \caption{Calculate N-th term of the Fibonacci sequence using iterative DP (Bottom Up)}
  \begin{algorithmic}[1]
    \Statex
    \Function{Fibonacci}{$n$}
        \State $fib[1] \gets 1$
        \State $fib[2] \gets 1$
        \For {$i = 3 \To n$}
            \State $fib[i] \gets fib[i - 2] + fib[i - 1]$
        \EndFor
        \Statex
        \State \Return $fib[n]$
    \EndFunction
  \end{algorithmic}
  
\end{algorithm}
\begin{algorithm}

  \caption{Calculate N-th term of the Fibonacci sequence using recursive DP (Top Down)}
  \begin{algorithmic}[1]
    \Ensure One Based Indexing for $fib$ array
    \State $fib \gets [1, 1, 0, 0, 0, 0, \dots]$
    \Statex
    \Function{Fibonacci}{$n$}
        \If {$fib[n] \neq 0$}
            \State \Return $fib[n]$
        \EndIf
        \State $fib[n] \gets \textbf{Fibonacci(n - 2)} + \textbf{Fibonacci(n - 1)}$
        \State \Return $fib[n]$
    \EndFunction
  \end{algorithmic}
\end{algorithm}

\end{document}
